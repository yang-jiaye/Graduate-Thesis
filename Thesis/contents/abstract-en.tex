\newpage
\addcontentsline{toc}{section}{ABSTRACT}
\setmainfont{Arial}
\section*{ABSTRACT}
%ABSTRCT:Arial 16磅加粗居中,单倍行距,段前24磅,段后18磅
\setmainfont{TeX Gyre Termes}

In the process of autonomous driving, Vehicle-Road Integration System is a particularly important part. Vehicle-Road Integration System refers to the use of wireless networks to closely connect the vehicle end and the road end to realize information exchange and information sharing. At present, the navigation and positioning of vehicle mainly depends on the global satellite navigation system (GNSS). But the system is limited by satellite-related errors, propagation path related errors and receiver relevant errors, thus limiting the accuracy of vehicle positioning and speed measurement.

In response to this problem, we hope to equip roadside units with LIDARs and cameras. And positioning the vehicles on the road end by object tracking method based on the fusion of  
LIDAR and vision. Then the positioning results will be sent back to the vehicle end, combined with the GNSS positioning results on the road end, predicting the vehicle position based on optimization method. This approach will compensate the positioning errors of GNSS in fragile scenes, and achieve the purpose of navigation enhancement.

We hope to adopt EagerMOT as the object tracking method. Its performance is evaluated on KITTI dataset. We trained 2D and 3D object detectors on the DAIR-V2X public dataset released by Tsinghua University and Baidu Inc. The 2D object detector adopts YOLOv4 framework and the 3D object detector adopts PointRCNN framework. Finally, we propose a vehicle positioning compensation method based on factor gragh.\\
%英文摘要内容:Times New Roman 12磅,行距20磅段前段后0磅
~\\ 
\textbf{Key words}: GNSS,roadside unit, object tracking, object detection, factor graph, trusted navigation
%Keywords:Times New Roman 12磅,行距20磅, “key words” 两词加粗