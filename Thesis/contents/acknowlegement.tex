\newpage
\fancyhead[LH]{上海交通大学学位论文}
\fancyhead[RH]{致谢}

\addcontentsline{toc}{section}{\texorpdfstring{致\qquad 谢}{致谢}}
\section*{致\qquad 谢}

感谢我的指导教师张欣副研究员,在课题开展的过程中提供了细致、耐心的指导,也带我快速学习导航领域的基础知识。课题组的战兴群副院长也在百忙中为课题思路提供了指导。同课题组的袁文翰师兄,刘佳辉师兄,王士壮师兄,池澄师兄都给予过我帮助,帮助我了解研究方向的最新进展,理清研究思路。

感谢印子斐老师,带领我初入科研道路。在参加PRP项目时,印老师从原理到代码都提供了大量指导与帮助。老师治学严谨,也时常鞭策我们进步。

感谢给我带来过优秀课堂的众多教师。陈克应老师的数学分析课让我感受到了世界一流大学该有的授课水平与教学氛围,老师幽默风趣又不失数学严谨性的授课方式至今让我记忆犹新。徐海光老师的大学物理课别具一格,用看似出乎意料实则在情理之中的授课方式,让我们不拘泥于题目,而是关注物理学本身的哲学。刘玉琴老师和邵鹏洁老师的大学俄语课也对我帮助很大。俄语是一门很难的学科,两位老师对我们的提问总是不吝啬自己的学识。如果没有两位老师负责认真的态度,我来到莫斯科不会有和当地人对话的基础。张峰老师的电路理论和电路实验课更是国家精品课程,我第一次认识到了“电路之美”。四年里还遇到了许多优秀老师,他们或者授课严谨,或者风格独特,一时难以写完。没有他们精彩的课堂,过去的四年将失去许多颜色。

致远工科的老师和同学是非常可爱的群体。师生间的融洽氛围让我第一次对交大有了归属感。在这里老师组织了许多有意思的活动,即使在20年疫情居家隔离期间,也组织了线上的音乐会。在这里,我认识很多有趣,聪明的小伙伴,他们勤勉的学习态度让我受益良多。回忆起来,在致远学习的时间仍然是四年里最充实最怀念的时间。

感谢航空航天学院对我们的关怀。大学是一个人走向社会的缓冲期,学院积极组织暑期实习,实地参观,但我们认识行业一线,逐渐认清自己未来的道路。学院还提供了来莫斯科交流的机会。第一次独立生活让我走出象牙塔,感受到了生活的烟火气。在留学生宿舍里,我感受到了世界的参差。

感谢我的舍友,在我失意的时候帮助我走出阴影,逐渐认识自我。

感谢我的家人,在我求学期间提供了温暖和关怀。