\newpage
\pagenumbering{Roman}
\fancyhead[LH]{上海交通大学学位论文}

\addcontentsline{toc}{section}{\texorpdfstring{摘\quad 要}{摘要}}
\section*{摘\quad 要}
%摘要:二字间空一格,黑体16磅加粗居中,单倍行距,段前24磅,段后18磅。

\hspace{8mm}

在自动驾驶的落地过程中,车路一体化系统是尤为重要的一环。车路一体化是指利用无线网络,将车端与路端紧密相连,实现车端与路端的信息交换、信息共享。目前的车路一体化或车路协同至多实现超视距态势通知,相关信息没有进入车辆自主导航的闭环可回信路,未能对保障车辆全天候、全场景下的导航,即目前导航领域最关心的“可信导航”有所贡献。本研究力求突破这一瓶颈。目前,车辆终端导航定位主要依赖于全球卫星导航系统(GNSS),但该系统受限于卫星相关误差、传播途径相关误差、接收机相关误差等限制,对车辆的定位、测速精度有限。\par

针对这一问题,我们希望在路侧单元装备激光雷达与相机,通过基于激光雷达与视觉融合的目标跟踪方法,完成车辆的路侧定位。然后将定位结果发回车端,结合车端的GNSS定位结果,通过基于优化的方法得到车辆位置的预测值,补偿GNSS在脆弱场景下的定位误差,达到导航增强的目的。\par

我们希望采用EagerMOT作为目标跟踪方法,并在KITTI\cite{geiger2013vision}数据集上验证了其效果。我们在由清华大学与百度公司发布的DAIR-V2X公开数据集上训练了二维与三维目标检测器。其中,二维目标检测器采用了YOLOv4框架,三维目标检测器采用了PointRCNN框架。最后我们提出了基于因子图优化的车辆定位补偿方法。\\

~\\
\textbf{关键词}:GNSS,新型路侧单元, 目标跟踪, 目标检测, 因子图,可信导航\\
%关键字:宋体12磅,行距20磅,段前段后0磅,关键字之间用逗号隔开,关键词三个字加粗。