\newpage
\fancyhead[LH]{上海交通大学学位论文}
\fancyhead[RH]{第六章\quad 全文总结}
\section{全文总结}
\subsection{全文工作总结}
在自动驾驶的落地过程中,车路一体化系统是尤为重要的一环。目前,车辆终端导航定位主要依赖于全球卫星导航系统(GNSS),但该系统受限于卫星相关误差、传播途径相关误差、接收机相关误差等限制,对车辆的定位、测速精度有限。这给车辆自身的定位能力带来了挑战。为了在GNSS脆弱环境下增强车辆的导航能力,本文对基于多传感器的路侧导航增强原理进行了研究。针对如何提高车辆对自身位置的估计精度,展开了以下研究。

首先调研了车端和路端框架下的公开数据集,总结了近年来适用于自动驾驶与车路协同的公开数据集。再调研了KITTI数据集和DAIR-V2X数据集的传感器配置,标签与标定格式,其中包括了激光雷达与摄像头的标定原理及其坐标转换关系。其次,在KITTI数据集上验证了EagerMOT跟踪算法,HOTA到达了78\%。然后调研了YOLOv4和PointRCNN的原理,在DAIR-V2X数据集上训练了2D与3D检测器。2D检测器的精度达到了95\%,3D检测器包围框精度达到了60\%。最后提出了车辆导航增强原理,先给出了问题的数学阐述,然后基于因子图进行建模,转化为了非线性最优化问题,最后介绍了常用的非线性最优化算法。

\subsection{创新点}

在文献调研中,我们发现虽然基于车端视角或路端视角的车辆跟踪框架有很多。但是鲜有将跟踪结果发挥车端,和车端自身定位结合估计车辆位置的工作。我们基于前人比较成熟的目标检测,目标跟踪工作,搭建好路侧跟踪框架后,提出了基于路侧单元的导航增强原理。

\subsection{不足与展望}

我们采用的路侧单元数据集只适合目标检测任务。但就在最近1个月,该团队发布了适合目标跟踪任务的数据集。我们最后也和另一数据集IPS+的作者取得了联系,获得了数据集使用授权。但此时工作进度已接近尾声,我们没有在更加合适的数据集上验证目标跟踪算法的性能。

此外,受到计算资源的限制,我们使用的检测器也比较落后了,YOLO系列已经迭代到了第7代,点云检测模型最近也涌现了许多更好的工作。

受到实验环境的限制,硬件平台尚未搭建完成,我们没有完成最后的路侧导航增强算法的实验部分,仅给出了理论推导。

未来进一步的工作是在最新的数据集上,尝试用不同检测模型训练,最后验证跟踪框架的性能。我们还计划在硬件平台完成后进行实车实验,收集数据验证导航增强原理的效果。