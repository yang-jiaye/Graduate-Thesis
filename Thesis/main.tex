\documentclass[UTF8,a4paper,12pt]{ctexart}

\usepackage{amsmath}
\numberwithin{equation}{section}
\allowdisplaybreaks[4]       %多行公式中换页
\usepackage{array}
\usepackage[font=small,font=bf,labelsep=none]{caption}
\usepackage{amssymb}
\usepackage{tikz}
\usepackage{amsthm}
\usepackage{mathrsfs}

\usepackage{dutchcal}
\usepackage{color}
\usepackage{graphicx}    %插入图片 
\usepackage{times}
\usepackage{mathptmx}
\usepackage{fancyhdr} %页眉页脚
\usepackage{booktabs}  %三线表



\pagestyle{fancy}
\fancyhf{}
\fancyfoot[C]{\thepage}
\usepackage{setspace}
\setlength{\baselineskip}{20pt}
\newcommand*{\circled}[1]{\lower.7ex\hbox{\tikz\draw (0pt, 0pt)%
    circle (.5em) node {\makebox[1em][c]{\small #1}};}}
\usepackage{hyperref}  %目录
\hypersetup{colorlinks=true,linkcolor=black}
\renewcommand {\thefigure} {\thesection{}-\arabic{figure}}%设定图片的编号。这样设置的实现效果为图1-1
\renewcommand {\thetable} {\thesection{}-\arabic{figure}}
\usepackage{caption}
\captionsetup{font={small},labelsep=quad}%文字5号,之间空一个汉字符位。
\captionsetup[table]{font={bf}} %表格表号与表题加粗
\captionsetup[figure]{font={bf}} %图号与标题加粗
\usepackage{appendix}
\usepackage{tocloft} 
\renewcommand{\cftsecleader}{\cftdotfill{\cftdotsep}} %为目录中section补上引导点
\usepackage{titletoc}
\titlecontents{section}[0pt]{\addvspace{6pt}\filright\bf}%
               {\contentspush{\thecontentslabel \quad}}%
               {}{\titlerule*[8pt]{.}\contentspage}
\makeatletter %双线页眉
\def\headrule{{\if@fancyplain\let\headrulewidth\plainheadrulewidth\fi%
\hrule\@height 1.5pt \@width\headwidth\vskip1.5pt%上面线为1pt粗
\hrule\@height 0.5pt\@width\headwidth  %下面0.5pt粗
\vskip-2\headrulewidth\vskip-1pt}      %两条线的距离1pt
  \vspace{6mm}}     %双线与下面正文之间的垂直间距
\makeatother

\ctexset { section = { format={\heiti \zihao {3} \bfseries \center } } }
\ctexset { section = { number={第\chinese {section}章} } } 

\usepackage[explicit]{titlesec}
\titlespacing*{\section}{0pt}{24pt plus .24pt minus .24pt}{18pt plus .0ex}

\setlength{\headheight}{14.48167pt} 
\setlength{\voffset}{-1.14cm}
\setlength{\topmargin}{0cm}
\setlength{\headsep}{2.9cm}
\hbadness=10000

\begin{document}

\thispagestyle{empty}

\renewcommand{\headrulewidth}{0pt}
\begin{figure}[htb] 
 \center{\includegraphics[width=5cm]  {fig1.png}} 
 \end{figure}

\begin{center}
\songti \zihao{-2} 上海交通大学学位论文
\end{center}
%该页为中文扉页。无需页眉页脚,纸质论文应装订在右侧
~\\
\begin{center}
\songti \zihao{1} \textbf{一种基于多源传感器信息融合的路侧导航增强单元}
\end{center}
%中文论文标题,1行或2行,宋体,加粗,二号,居中。论文题目不得超过36个汉字
~\\
~\\
~\\
\begin{center}
\heiti \zihao{4}
\begin{tabular}{r@{:}l}
\textbf{姓\quad 名} & \textbf{杨嘉业} \\
\textbf{学\quad 号} & \textbf{519021910359} \\
\textbf{导\quad 师} & \textbf{张欣} \\
\textbf{学\quad 院} & \textbf{航空航天学院} \\
\textbf{学科/专业名称} & \textbf{航空航天工程} \\
\textbf{申请学位层次} & \textbf{学士学位} \\
\end{tabular}
\end{center}
~\\
\begin{center}
\songti \zihao{4} \textbf{2023年5月}
\end{center}

\newpage
\thispagestyle{empty}
~\\
\begin{center}
\zihao{4}
\textbf{
A Dissertation Submitted to \\
Shanghai Jiao Tong University for Bachelor Degree}
\end{center}
~\\
\begin{center}
\zihao{-2}\textbf{
A ROADSIDE UNIT FOR NAVIGATION ENHANCEMENT BASED ON MULTI-SENSORS INFORMATION FUSION}
\end{center}
%英文论文标题:大写,Times New Roman,加粗,14 points,居中
~\\
~\\
~\\
\begin{center}
\zihao{3}
Author: Yang Jiaye\\
Supervisor: Zhang Xin
\end{center}
~\\
~\\
~\\
\begin{center}
\zihao{3}
School of Aeronautics and Astronautics \\
Shanghai Jiao Tong University \\
Shanghai, P.R.China \\
May 14$^{\mathrm{th}}$, 2023 
\end{center}

\newpage
\thispagestyle{empty}
\begin{center}
\heiti \zihao{3}\textbf{
上海交通大学\\
学位论文原创性声明}
\end{center}

\zihao{-4}
本人郑重声明:所呈交的学位论文,是本人在导师的指导下,独立进行研究工作所取得的成果。除文中已经注明引用的内容外,本论文不包含任何其他个人或集体已经发表或撰写过的作品成果。对本文的研究做出重要贡献的个人和集体,均已在文中以明确方式标明。本人完全知晓本声明的法律后果由本人承担。

\begin{flushright}
\begin{tabular}{l}
\zihao{4}
学位论文作者签名:\hspace{20mm}\qquad\\
\zihao{4}
日期:\qquad 年\qquad 月\qquad 日
\end{tabular}
\end{flushright}

~\\
\begin{center}
\heiti \zihao{3}\textbf{
上海交通大学\\
学位论文使用授权书}
\end{center}

本人同意学校保留并向国家有关部门或机构送交论文的复印件和电子版,允许论文被查阅和借阅。\\
本学位论文属于 :\par
□公开论文\par
□内部论文,保密□1年/□2年/□3年,过保密期后适用本授权书。\par
□秘密论文,保密\_\_\_年(不超过10年),过保密期后适用本授权书。\par
□机密论文,保密\_\_\_年(不超过20年),过保密期后适用本授权书。\par
(请在以上方框内选择打“√”)\\

\begin{flushright}
\zihao{4}
\begin{tabular}{l l}
学位论文作者签名:\hspace{10mm}\qquad \hspace{100mm}&指导教师签名:\qquad \\
日期:\qquad 年\qquad 月\qquad 日 &日期:\qquad 年\qquad 月\qquad 日\\
\end{tabular}
\end{flushright}


%%%%%%%%%%%%%%%%%%%%%%%%%%%%%%%%%%%%%%%%%%%%%%%%%%%%%%%%%%%%%%%%%%%%%%%%%%%%%%%
%摘要页
%%%%%%%%%%%%%%%%%%%%%%%%%%%%%%%%%%%%%%%%%%%%%%%%%%%%%%%%%%%%%%%%%%%%%%%%%%%%%%%


\newpage
\pagenumbering{Roman}
\fancyhead[LH]{上海交通大学学位论文}

\addcontentsline{toc}{section}{\texorpdfstring{摘\quad 要}{摘要}}
\section*{摘\quad 要}
%摘要:二字间空一格,黑体16磅加粗居中,单倍行距,段前24磅,段后18磅。

\hspace{8mm}

“十三五”期间我国综合交通运输发展取得了显著成效,但与经济社会高质量发展的总体要求相比,仍存在智慧交通发展水平不高等问题。随着北斗三号全球卫星导航系统等核心空间基础设施的开通服务、新一代信息技术的发展,2022国家重点研发计划《广域交通可信导航信号与时空服务系统关键技术》应运而生。本课题主要的研究内容响应了该项目的课题3“高精泛源时空感知网络及车路一体化信息融合技术”。\par
在自动驾驶的落地过程中,车路一体化系统是尤为重要的一环。车路一体化是指利用无线网络,将车端与路端紧密相连,实现车端与路端的信息交换、信息共享。目前,车辆终端导航定位主要依赖于全球卫星导航系统(GNSS),但该系统受限于卫星相关误差、传播途径相关误差、接收机相关误差等限制,对车辆的定位、测速精度有限。\par
针对这一问题
~\\
\textbf{关键词}:学位论文,论文格式,规范化,模板\\
%关键字:宋体12磅,行距20磅,段前段后0磅,关键字之间用逗号隔开,关键词三个字加粗。


%%%%%%%%%%%%%%%%%%%%%%%%%%%%%%%%%%%%%%%%%%%%%%%%%%%%%%%%%%%%%%%%%%%%%%%%%%%%%%%
%Abstract
%%%%%%%%%%%%%%%%%%%%%%%%%%%%%%%%%%%%%%%%%%%%%%%%%%%%%%%%%%%%%%%%%%%%%%%%%%%%%%%


\newpage
\addcontentsline{toc}{section}{ABSTRACT}
\section*{ABSTRACT}
%ABSTRCT:Arial 16磅加粗居中,单倍行距,段前24磅,段后18磅

\hspace{8mm}As a primary means of demonstrating research findings for undergraduate students, dissertation is a systematic and standardized record of the new inventions, theories or insights obtained by the author in the research work. It can not only function as an important reference when students pursue further studies, but also contribute to scientific research and social development.\par 
This template is therefore made to improve the quality of undergraduates’ dissertation and to further standardize it both in content and in format.\\
%英文摘要内容:Times New Roman 12磅,行距20磅段前段后0磅
~\\ 
\textbf{Key words}: dissertation, dissertation format, standardization, template
%Keywords:Times New Roman 12磅,行距20磅, “key words” 两词加粗


%%%%%%%%%%%%%%%%%%%%%%%%%%%%%%%%%%%%%%%%%%%%%%%%%%%%%%%%%%%%%%%%%%%%%%%%%%%%%%%
%目录
%%%%%%%%%%%%%%%%%%%%%%%%%%%%%%%%%%%%%%%%%%%%%%%%%%%%%%%%%%%%%%%%%%%%%%%%%%%%%%%


\newpage
\renewcommand\contentsname{\textbf{目\quad 录}}
\begin{center}
{\tableofcontents
\thispagestyle{fancy}
\fancyhead [LO, R] {\normalsize{\songti 第一章\quad 绪论}}
%\fancyhead [LO, R] {\normalsize{\songti 上海交通大学学位论文}}
}
\end{center}


%%%%%%%%%%%%%%%%%%%%%%%%%%%%%%%%%%%%%%%%%%%%%%%%%%%%%%%%%%%%%%%%%%%%%%%%%%%%%%%
%第一章 绪论
%%%%%%%%%%%%%%%%%%%%%%%%%%%%%%%%%%%%%%%%%%%%%%%%%%%%%%%%%%%%%%%%%%%%%%%%%%%%%%%


\newpage
\fancyhead[LH]{上海交通大学学位论文}
\fancyhead[RH]{第一章\quad 绪论}
\pagenumbering{arabic}
\section{绪论}
\subsection{引言}
学位论文……
\subsection{本文主要研究内容}
本文……
\subsection{本文研究意义}
本文……
\subsection{本章小结}
本文……


%%%%%%%%%%%%%%%%%%%%%%%%%%%%%%%%%%%%%%%%%%%%%%%%%%%%%%%%%%%%%%%%%%%%%%%%%%%%%%%
%第二章 数据集框架
%%%%%%%%%%%%%%%%%%%%%%%%%%%%%%%%%%%%%%%%%%%%%%%%%%%%%%%%%%%%%%%%%%%%%%%%%%%%%%%


\newpage
\fancyhead[LH]{上海交通大学学位论文}
\fancyhead[RH]{第二章\quad 数据集}
\pagenumbering{arabic}
\section{数据集}

\subsection{相机雷达联合标定}



\subsection{本章小结}
本文……

%%%%%%%%%%%%%%%%%%%%%%%%%%%%%%%%%%%%%%%%%%%%%%%%%%%%%%%%%%%%%%%%%%%%%%%%%%%%%%%
%第三章 目标检测器训练 
%%%%%%%%%%%%%%%%%%%%%%%%%%%%%%%%%%%%%%%%%%%%%%%%%%%%%%%%%%%%%%%%%%%%%%%%%%%%%%%


\newpage
\fancyhead[LH]{上海交通大学学位论文}
\fancyhead[RH]{第二章\quad 数据集}
\pagenumbering{arabic}
\section{数据集}

\subsection{相机雷达联合标定}

%%%%%%%%%%%%%%%%%%%%%%%%%%%%%%%%%%%%%%%%%%%%%%%%%%%%%%%%%%%%%%%%%%%%%%%%%%%%%%%
%第四章 目标跟踪器搭建
%%%%%%%%%%%%%%%%%%%%%%%%%%%%%%%%%%%%%%%%%%%%%%%%%%%%%%%%%%%%%%%%%%%%%%%%%%%%%%%


\newpage
\fancyhead[LH]{上海交通大学学位论文}
\fancyhead[RH]{第二章\quad 数据集}
\pagenumbering{arabic}
\section{数据集}

%%%%%%%%%%%%%%%%%%%%%%%%%%%%%%%%%%%%%%%%%%%%%%%%%%%%%%%%%%%%%%%%%%%%%%%%%%%%%%%
%第五章 基于优化的路侧增强原理
%%%%%%%%%%%%%%%%%%%%%%%%%%%%%%%%%%%%%%%%%%%%%%%%%%%%%%%%%%%%%%%%%%%%%%%%%%%%%%%


\newpage
\fancyhead[LH]{上海交通大学学位论文}
\fancyhead[RH]{第五章\quad 基于优化的路侧增强原理}
\pagenumbering{arabic}
\section{基于优化的路侧增强原理}

\subsection{因子图及其增量非线性最优化方法}
% gtsam 


\subsection{基于路侧增强的车辆可靠位置推理}


\subsection{本章小结}
本文……

\subsection{本章小结}
本文……

\subsection{本章小结}
本文……


%%%%%%%%%%%%%%%%%%%%%%%%%%%%%%%%%%%%%%%%%%%%%%%%%%%%%%%%%%%%%%%%%%%%%%%%%%%%%%%
%参考文献
%%%%%%%%%%%%%%%%%%%%%%%%%%%%%%%%%%%%%%%%%%%%%%%%%%%%%%%%%%%%%%%%%%%%%%%%%%%%%%%

\newpage
\fancyhead[LH]{上海交通大学学位论文}
\fancyhead[RH]{参考文献}

\addcontentsline{toc}{section}{\texorpdfstring{参\quad 考\quad 文\quad 献}{参考文献}}
\renewcommand\refname{参\quad 考\quad 文\quad 献}
\begin{thebibliography}{1}
\bibitem{1} 杨瑞林, 李力军. 新型低合金高强韧性耐磨钢的研究[J]. 钢铁. 1999(7): 41-45.
\bibitem{2} 于潇, 刘义, 柴跃廷, 等. 互联网药品可信交易环境中主体资质审核备案模式[J]. 清华大学学报(自然科学版), 2012, 52(11): 1518-1523.
\bibitem{3} Schinstock D.E., Cuttino J.F. Real time kinematic solutions of a non-contacting, three dimensional metrology frame[J]. Precision Engineering. 2000, 24(1): 70-76. 
\bibitem{4} 温诗铸. 摩擦学原理[M]. 北京: 清华大学出版社, 1990: 296-300.
\bibitem{5} 蒋有绪, 郭泉水, 马娟, 等. 中国森林群落分类及其群落学特征[M]. 北京: 科学出版社, 1998: 5-17.
\bibitem{6} 贾名字. 工程硕士论文撰写规范[D]. 上海: 上海交通大学, 2000: 177-178.
\bibitem{7} 张凯军. 轨道火车及高速轨道火车紧急安全制动辅助装置: 201220158825.2[P]. 2012-04-05.
\bibitem{8} 全国信息与文献标准化技术委员会. 文献著录: 第4部分 非书资料: GB/T 3792.4-2009[S]. 北京: 中国标准出版社, 2010: 3.
\end{thebibliography}%(参考文献格式请参考GB/T 7714-2015《信息与文献 参考文献著录规则》)

\newpage
\fancyhead[LH]{上海交通大学学位论文}
\fancyhead[RH]{附录1}

\addcontentsline{toc}{section}{附录}
\section*{符号与标记(附录1)}

\newpage
\fancyhead[LH]{上海交通大学学位论文}
\fancyhead[RH]{学术论文和科研成果目录}

\addcontentsline{toc}{section}{攻读学位期间学术论文和科研成果目录}
\section*{攻读学位期间学术论文和科研成果目录}

[1] 张三,李四. …… (已录用)

\newpage
\fancyhead[LH]{上海交通大学学位论文}
\fancyhead[RH]{致谢}

\addcontentsline{toc}{section}{\texorpdfstring{致\qquad 谢}{致谢}}
\section*{致\qquad 谢}

\hspace{8mm}致谢主要感谢导师和对论文工作有直接贡献和帮助的人士和单位。致谢言语应谦虚诚恳,实事求是。


\newpage
\pagenumbering{arabic}
\fancyhead[LH]{上海交通大学学位论文}
\fancyhead[RH]{}
\section*{NUMERICAL SIMULATION OF HOMOGENEOUS CHARGE COMPRESSION IGNITION COMBUSTION FUELED WITH DIMETHYL ETHER}%英文大摘要标题

\hspace{8mm}HCCI (Homogenous Charge Compression Ignition) combustion has advantages in terms of efficiency and reduced emission. HCCI combustion can not only ensure both the high economic and dynamic quality of the engine, but also efficiently reduce the NOx and smoke emission. Moreover, one of the remarkable characteristics of HCCI combustion is that the ignition and combustion process are controlled by the chemical kinetics, so the HCCI ignition time can vary significantly with the changes of engine configuration parameters and operating conditions. ……%(英文大摘要正文)



\end{document} 